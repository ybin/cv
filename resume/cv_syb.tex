% !TEX TS-program = xelatex
% !TEX encoding = UTF-8 Unicode
% !Mode:: "TeX:UTF-8"

\documentclass{resume}
\usepackage{zh_CN-Adobefonts_external} % Simplified Chinese Support using external fonts (./fonts/zh_CN-Adobe/)
%\usepackage{zh_CN-Adobefonts_internal} % Simplified Chinese Support using system fonts
\usepackage{linespacing_fix} % disable extra space before next section
\usepackage{cite}

\begin{document}
\pagenumbering{gobble} % suppress displaying page number

\name{孙延宾}

% {E-mail}{mobilephone}{homepage}
% be careful of _ in emaill address
\contactInfo{ybin.sun@163.com}{(+86) 187 0671 8156}{http://ybin.cc/}
% {E-mail}{mobilephone}
% keep the last empty braces!
%\contactInfo{xxx@yuanbin.me}{(+86) 131-221-87xxx}{}

\section{\faGraduationCap\  教育经历}
\datedsubsection{\textbf{中南大学}, 湖南, 长沙}{2008 -- 2011}
\textit{硕士}\ 基础数学
\datedsubsection{\textbf{山东大学(威海)}, 山东, 威海}{2004 -- 2008}
\textit{学士}\ 信息与计算科学

\section{\faUsers\ 工作经历}
\datedsubsection{\textbf{中兴通讯}(西安)}{2011年7月 -- 至今}
\role{手机应用软件开发高级工程师}{}
\begin{onehalfspacing}
\begin{itemize}
  \item 共同完成Camera APP新架构的设计和开发
  \item 带领团队顺利完成Camera API的迁移以及Android Studio开发环境的转换
  \item 承担应用代码版本控制管理、自动化测试以及持续集成等工作任务
  \item 使用OpenGL ES、并行计算以及ARM NEON技术加速图像处理
  \item 带领团队完成ZTE手机Camera应用的开发及维护工作
\end{itemize}
\end{onehalfspacing}


% Reference Test
%\datedsubsection{\textbf{Paper Title\cite{zaharia2012resilient}}}{May. 2015}
%An xxx optimized for xxx\cite{verma2015large}
%\begin{itemize}
%  \item main contribution
%\end{itemize}

\section{\faCogs\ Android技能}
% increase linespacing [parsep=0.5ex]
\begin{description}[parsep=0.5ex]
  \item[Android系统] \
    \begin{description}
         \item[应用开发] \ \\ 熟练掌握Android App开发流程、架构设计。
            设计并实现ZTE Camera应用,包括独立的UI系统和动画效果、第三方算法库的集成方案、Camera API兼容方法、OpenGL ES和RenderScript加速图像处理等。
         \item[View渲染流程] \ \\ 理解Android View的渲染流程(measure, layout, draw),完成自定义UI组件。
         \item[Android消息机制] \ \\ 理解Android Binder机制以及service manger,能编写native service和system service,了解binder driver的实现,利用该技术实现JPEG编码的硬件加速。
            理解Android的handler消息处理机制、Android application与Linux process之间的关系。
         \item [Android Graphic System] \ \\ 深入了解Android图形系统,以及利用OpenGL实现高效的图像显示。
         \item [Camera技术栈] 深入理解Android平台的Camera技术栈,APP业务流程、Framework组织架构及实现、新的HAL模块化机制,
            了解高通Camera HAL实现机制。
         \item[整体架构] \ \\ 了解Android的整体架构,包括Linux kernel,Android native service
            (如service manager, activity manager, window manager etc.) 等。
            了解从Linux kernel到native service到Zygote到system service的整个流程,并阅读过相关源码。
    \end{description}
    
  \item[开发工具] \
    \begin{itemize}
        \item 熟练使用ADT、Android Studio开发环境和Gradle构建系统,能够开发插件帮助业务开发。
        \item 熟练使用Vim、Emacs文本编辑器
    \end{itemize}
\end{description}

\newpage

\section{\faCogs\ 编程技能}
\begin{description}[parsep=0.5ex]
  \item[JVM平台] \ \\ 理解Java虚拟机平台,熟悉Java class文件结构、类加载机制,JVM运行时内存、heap组成结构以及垃圾回收等,为学习JVM,曾仔细阅读JamVM(Dalvik VM的前身)源码,学习Java虚拟机的实现。
  \item[Java编程] \ \\ 熟练掌握面向对象编程,能熟练使用Java的各种collections、范型、并发以及Lambda编程。熟练掌握Java多态、内部类及初始化流程。
  \item[Kotlin编程] \ \\ 熟练使用Kotlin进行Android APP开发,掌握Kotlin语法、扩展函数、函数式编程以及协程,并理解这些语法在Java字节码层面的实现原理。
  \item[C、C++以及JNI编程] \ \\ 熟练掌握C语言编程,理解编译、链接的过程; 了解C++面向对象、多重继承、类模板等,阅读代码无压力。了解ARM NEON技术并利用该技术加速图像处理。
  \item[函数式编程] \ \\ 能清晰的理解函数的一等公民特性、数据不可变性、纯函数特性、闭包、惰性求值等,学习过Clojure编程,了解Emacs lisp语言、Groovy语言。
  \item[OpenGL以及并行计算] \ \\ 熟练使用EGL和OpenGL ES的API,理解其概念以及开发流程,使用该技术实现Camera预览的实时滤镜效果。在Android平台使用RenderScript进行并行计算,完成图像颜色空间转换以及水印效果。
  \item[机器学习] \ \\了解经典的机器学习算法(决策树、贝叶斯网络等),熟悉深度学习及其常见的神经网络(CNN、RNN、LSTM等),熟练使用Python及Tensorflow深度学习框架,参与实际项目利用该技术实现实时人像肤色检测。
  \item[Linux平台] \
    \begin{description}
      \item[Kernel] \ \\ 了解内存管理、虚拟文件系统、驱动程序编写,熟悉进程的内存布局,曾深入内核源码进行部分学习,完整阅读过XV6项目的源码。
      \item[Linux环境] \ \\ 熟悉Linux操作系统,日常使用以及工作中熟练使用Ubuntu系统。
      \item[Shell环境] \ \\ 了解Shell编程,纯命令行下工作无压力。
    \end{description}
\end{description}

\section{\faInfo\ 其他}
% increase linespacing [parsep=0.5ex]
\begin{description}
  \item[英语] \ \\ 英语六级。
  \item[Git] \ \\ 熟练使用并且透彻理解git的内部原理,工作以及自己业余写的代码都会用git来进行版本控制。
  \item[Gradle以及CMake编译系统] \ \\ 熟练使用Gradle编译系统进行APP持续集成以及自动进行各个项目之间的适配;
    使用CMake系统进行JNI代码编译及第三方算法库的配置管理。
  \item[\LaTeX{}排版系统] \ \\ 熟练使用,毕业论文、技术交流使用的文档、公司培训使用的幻灯片都是用\LaTeX{}完成的。
  \item[专利] \ \\ 《一种目标屏幕确定方法、装置及存储介质》
\end{description}



\section{\faInfo\ Introduction}
I'm Eric, 33 years old, born in Shandong Province.

I graduated from Shandong University at Weihai, a small coastal city in eastern Shandong. And I got my bachelor degree after my graduation. Although my major is Mathematics, I was also interested in computer science and learned some courses by myself.

After that, I went to Central South Univerty, and got my master degree after graduation. I did not want to be a math teacher, but was interested in programming, so I found a job at ZTE corporation to work on Android Applications.

From then on, I worked on the camera application of ZTE phones. It was almost seven years. I was a mobile software developer and became the team leader since last year.

I like programming, but I don't want to do the same programming. I want to change, so I come here.


%% Reference
%\newpage
%\bibliographystyle{IEEETran}
%\bibliography{mycite}
\end{document}
