% !TEX TS-program = xelatex
% !TEX encoding = UTF-8 Unicode
% !Mode:: "TeX:UTF-8"

\documentclass{resume}
\usepackage{zh_CN-Adobefonts_external}
\usepackage{linespacing_fix}
\usepackage{cite}

\begin{document}
\pagenumbering{gobble} % suppress displaying page number

\name{胡星华}
\contactInfo{xiaxing.hu@qq.com}{(+86) 182 9288 1595}{}

\section{\faGraduationCap\  教育经历}
\datedsubsection{\textbf{湖南大学}, 湖南, 长沙}{2008 -- 2011}
\textit{硕士}\ 计算机科学与技术
\datedsubsection{\textbf{鲁东大学}, 山东, 烟台}{2004 -- 2008}
\textit{学士}\ 数学与应用数学

\vspace{1em}

\section{\faUsers\ 工作经历}
%increase linespacing [parsep=0.5ex]
\datedsubsection{\textbf{中兴通讯}(西安)}{2011年7月 -- 至今}
\role{软件开发工程师}{}

\begin{description}[parsep=1em]
  \item[2016年6月 -- 现在]\hfill
    \begin{itemize}[itemsep=1ex,topsep=-1ex,leftmargin=0em]
      \item 作为调度领域BA,负责研讨上行调度和atu的需求,被评为需求交付女王
      \item 熟悉cmac上行调度框架,ulharq,上行rb预估,ulrrm,gap,上行预调度,
      ping包优化,atu,speedtest、smarttest等测速软件优化,异系统下lte共享rb资源的优化处理,做过以上模块需求开发,并如期保质交付
      \item 对drx、dlharq、ulsumimo、配比0、ulmumimo、ulqos、
      mac主控、航线有初步了解
      \item 可以运用MFQ原则对需求设计用例和在vs上进行eft自测
      \item 有一定的团队管理经验,作为代理SM管理过团队一段时间,团队可以有条不紊的运作
      \item 多次交付日本紧急需求开发、中移招标和联通招标测开发,并出差去前方支持测试,保证测试顺利完成
    \end{itemize}
  
  \item[2015年4月 -- 2016年5月]\hfill\\
    加入atu特性团队,负责ulcmac代码优化工作,优化后上行atu流量提升10\%。
    
  \item[2014年10月 -- 2015年3月]\hfill\\
    作为上行cmac调度一对实验室故障接口人,有效做好团队的防火墙工作。
    
  \item[2014年4月 -- 2014年9月]\hfill\\
    负责上行cmac调度一队相关故事的开发、自测、联调以及代码合入工作(prb随机化、magic radio、
    speedtest、中移延迟调度、中移配比0下mumimo、中移射频测试等)。

  \item[2013年10月 -- 2014年3月]\hfill\\
    休产假。
      
  \item[2012年10月 -- 2013年9月]\hfill\\
    3.2版本ulcmac航线高铁需求研讨、代码实现、功能自测、联调入主版本、测试部测试、外场测试、
    定位解决测试部以及外场故障。
  
  \item[2011年7月 -- 2012年9月]\hfill\\
    熟悉环境搭建,进行100ue集成测试,简单功能开发(DCI0lost、ulharq、ulgap、timer、RLF等)。
    
\end{description}

\newpage

\section{\faInfo\ 其他}
% increase linespacing [parsep=0.5ex]
\begin{description}[parsep=1em]
  \item[C语言]\hfill\\
    工作中主要使用的编程语言
  
  \item[英语]\hfill\\
    英语六级
  
  \item[代码管理]\hfill\\
    熟练使用svn和git进行版本控制
  
  \item[个人考核]\hfill\\
  第一个半年考核为S,第二个半年考核由于公司原因取消,半数以上月考核为S,\\ 被评为部门月度之星
    
  \item[兴趣爱好]\hfill\\
    热爱生活,性格开朗,喜欢打羽毛球、跑步。
\end{description}

\end{document}
